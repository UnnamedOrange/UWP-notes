% MIT License

% Copyright (c) 2019 Orange Lee

% Permission is hereby granted, free of charge, to any person obtaining a copy
% of this software and associated documentation files (the "Software"), to deal
% in the Software without restriction, including without limitation the rights
% to use, copy, modify, merge, publish, distribute, sublicense, and/or sell
% copies of the Software, and to permit persons to whom the Software is
% furnished to do so, subject to the following conditions:

% The above copyright notice and this permission notice shall be included in all
% copies or substantial portions of the Software.

% THE SOFTWARE IS PROVIDED "AS IS", WITHOUT WARRANTY OF ANY KIND, EXPRESS OR
% IMPLIED, INCLUDING BUT NOT LIMITED TO THE WARRANTIES OF MERCHANTABILITY,
% FITNESS FOR A PARTICULAR PURPOSE AND NONINFRINGEMENT. IN NO EVENT SHALL THE
% AUTHORS OR COPYRIGHT HOLDERS BE LIABLE FOR ANY CLAIM, DAMAGES OR OTHER
% LIABILITY, WHETHER IN AN ACTION OF CONTRACT, TORT OR OTHERWISE, ARISING FROM,
% OUT OF OR IN CONNECTION WITH THE SOFTWARE OR THE USE OR OTHER DEALINGS IN THE
% SOFTWARE.

\section{XML 基础}

为什么是 XML 而不是 XAML 呢?因为正如前面提到的两者的概念,XAML 本身就是一种特殊的 XML。

\subsection{XML 的结构\cite{XMLrunoob1}\cite{XMLrunoob2}\cite{XMLrunoob3}}

XML 文档形成一棵有根树结构。一个 XML 文档\textbf{有且仅有}一个根。

XML 的根结点是一个\emph{元素}\index{XML!元素}。根结点的孩子也是\emph{元素},例如:
\begin{lstlisting}[language = xml]
<root>
    <child1></child1>
    <child2></child2>
</root>
\end{lstlisting}

一个元素可以像上面的代码一样包含多个元素。除此之外,一个元素还可以拥有一个或多个\emph{属性}\index{XML!属性}。如:
\begin{lstlisting}[language = xml]
<root name = "根儿" tips = "こんにちは!">
    <child1></child1>
</root>
\end{lstlisting}

\emph{属性值}\index{XML!属性值}\textbf{必须用引号括起来},但是却既可以使用单引号又可以使用双引号,这一点上与 Python 是一致的。

一个元素还可以包含\emph{文本}\index{XML!文本}。如:
\begin{lstlisting}[language = xml]
<root>
    <age>20</age>
    <name>Big Billy Bob</name>
</root>
\end{lstlisting}

文本不需要使用引号括起来,除非你想让引号成为数据的一部分。看到上面的代码,你可能会想:要是 Big Billy Bob 住在得克萨斯州(加上一句 \code{<address>Texas</address>}),外部程序还能正确处理吗?当然可以,因为它们只负责读取名为 \code{age} 和 \code{name} 的元素,\textbf{就算有了扩展},\code{age} 和 \code{name} 当然也还是能够被正常读取的。

\subsection{XML 的语法\cite{XMLrunoob2}}

\subsubsection{大小写敏感}

XML 是大小写敏感的。

\subsubsection{注释\index{XML!注释}}

XML 的\emph{注释}用 \code{<!--} 和 \code{-->} 包围,如:
\begin{lstlisting}[language = xml]
<!-- comment: 初めまして -->
\end{lstlisting}

\subsubsection{声明\index{XML!声明}}

在文档的第一行,可以加上以下代码:
\begin{lstlisting}[language = xml]
<?xml version="1.0" encoding="UTF-8"?>
\end{lstlisting}

\code{UTF-8} 是 XML 的默认编码,所以可以很放心地写道:XML,初めまして!

\subsubsection{打开标签与关闭标签\index{XML!打开标签}\index{XML!关闭标签}}

我们之前其实已经见过\emph{标签}\index{标签}了。在 XML 中,有一个打开标签(如 \code{<child>}),就一定有一个关闭标签(如 \code{</child>})。唯一的例外是,上面的\emph{声明}没有关闭标签。

在 XML 中,打开标签和关闭标签必须满足括号的排列规则,例如:
\begin{lstlisting}[language = xml]
<p1><p2></p2></p1> <!-- Right -->
<p1><p2></p1></p2> <!-- WRONG -->
\end{lstlisting}

XML 还可以使用\emph{自关闭标签}\index{XML!自关闭标签},例如:
\begin{lstlisting}[language = xml]
<p1 attr = "la啦ラ~~~"/>
\end{lstlisting}

\subsubsection{实体引用\index{XML!实体引用}}

如何在 XML 中输入 \code{<} 作为文本的一部分呢?XML 定义了 5 个\emph{实体引用}。实体引用类似于转义字符,见下表:
\begin{table}[htbp]
    \centering
    \begin{tabular}{|c|c|}
        \hline
        \code{<}  & \code{\&lt;} \\\hline
        \code{>}  & \code{\&gt;} \\\hline
        \code{\&} & \code{\&amp;} \\\hline
        \code{'}  & \code{\&apos;} \\\hline
        \code{"}  & \code{\&quot;} \\\hline
    \end{tabular}
    \caption{XML 实体引用一览}
\end{table}

\subsection{命名空间\index{XML!命名空间}\cite{XMLrunoob4}}

可以给 XML 指定一个\emph{命名空间(namespace)},然后通过一个\emph{前缀}访问这个命名空间。参看以下代码(原谅我直接抄了):
\begin{lstlisting}[language = xml]
<root>
    <h:table xmlns:h="http://www.w3.org/TR/html4/">
        <h:tr>
            <h:td>Apples</h:td>
            <h:td>Bananas</h:td>
        </h:tr>
    </h:table>
</root>
\end{lstlisting}

命名空间既可以在被使用的地方定义,也可以在根结点定义,其语法是:\code{xmlns:prefix = "URL"}。

命名空间的 URL 是为了给这个命名空间赋予一个独一无二的名称,但很多命名空间都指向一个网页,这个网页可能有些帮助信息。在 XML 文件使用时,这个 URL 不会被访问,仅作为一个标志。

\subsubsection{默认命名空间\index{XML!命名空间!默认命名空间}}

如果你不想使用前缀,但是又想使用命名空间,则可以使用\emph{默认命名空间}。见下例(原谅我直接抄了):
\begin{lstlisting}[language = xml]
<table xmlns="http://www.w3.org/TR/html4/">
    <tr>
        <td>Apples</td>
        <td>Bananas</td>
    </tr>
</table>
\end{lstlisting}