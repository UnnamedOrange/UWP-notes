% MIT License

% Copyright (c) 2019 Orange Lee

% Permission is hereby granted, free of charge, to any person obtaining a copy
% of this software and associated documentation files (the "Software"), to deal
% in the Software without restriction, including without limitation the rights
% to use, copy, modify, merge, publish, distribute, sublicense, and/or sell
% copies of the Software, and to permit persons to whom the Software is
% furnished to do so, subject to the following conditions:

% The above copyright notice and this permission notice shall be included in all
% copies or substantial portions of the Software.

% THE SOFTWARE IS PROVIDED "AS IS", WITHOUT WARRANTY OF ANY KIND, EXPRESS OR
% IMPLIED, INCLUDING BUT NOT LIMITED TO THE WARRANTIES OF MERCHANTABILITY,
% FITNESS FOR A PARTICULAR PURPOSE AND NONINFRINGEMENT. IN NO EVENT SHALL THE
% AUTHORS OR COPYRIGHT HOLDERS BE LIABLE FOR ANY CLAIM, DAMAGES OR OTHER
% LIABILITY, WHETHER IN AN ACTION OF CONTRACT, TORT OR OTHERWISE, ARISING FROM,
% OUT OF OR IN CONNECTION WITH THE SOFTWARE OR THE USE OR OTHER DEALINGS IN THE
% SOFTWARE.

\section{XML 基础}

为什么是 XML 而不是 XAML 呢?因为正如前面提到的两者的概念,XAML 本身就是一种特殊的 XML。

\subsection{XML 的结构\cite{XMLrunoob1}\cite{XMLrunoob2}\cite{XMLrunoob3}}

XML 文档形成一棵有根树结构。一个 XML 文档\textbf{有且仅有}一个根。

XML 的根结点是一个\emph{元素}\index{XML!元素}。根结点的孩子也是\emph{元素},例如:
\begin{lstlisting}[language = xml]
<root>
    <child1></child1>
    <child2></child2>
</root>
\end{lstlisting}

一个元素可以像上面的代码一样包含多个元素。除此之外,一个元素还可以拥有一个或多个\emph{属性}\index{XML!属性}。如:
\begin{lstlisting}[language = xml]
<root name = "根儿" tips = "こんにちは!">
    <child1></child1>
</root>
\end{lstlisting}

\emph{属性值}\index{XML!属性值}\textbf{必须用引号括起来},但是却既可以使用单引号又可以使用双引号,这一点上与 Python 是一致的。

一个元素还可以包含\emph{文本}\index{XML!文本}。如:
\begin{lstlisting}[language = xml]
<root>
    <age>20</age>
    <name>Big Bili Bob</name>
</root>
\end{lstlisting}

文本不需要使用引号括起来,除非你想让引号成为数据的一部分。看到上面的代码,你可能会想:要是 Big Bili Bob 住在得克萨斯州(加上一句 <address>Texas</address>),外部程序还能正确处理吗?当然可以,因为它们只负责读取名为 age 和 name 的元素,\textbf{就算有了扩展},age 和 name 当然也还是能够被正常读取的。

\subsection{XML 的语法\cite{XMLrunoob2}}
